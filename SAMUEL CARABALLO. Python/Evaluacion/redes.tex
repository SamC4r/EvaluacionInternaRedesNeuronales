\documentclass[12pt]{article}

% remove spacing around date:

\usepackage{setspace}

\author{Samuel Caraballo}

\title{Redes Neuronales\\
\large Evaluaci\'{o}n Interna
}

\date{\vspace{-5ex}} % Para quitar la fecha



\begin{document}
    \maketitle

    \section*{Introduccion}
    \begin{onehalfspace}
    
    
      El siguiente documento constituye un estudio sobre 
      Redes Neuronales. Asimismo, se estudiar\'{a}n las bases 
      matem\'{a}ticas a fin de comprender el funcionamiento de los
      algoritmos y, un ejemplos de c\'{o}digo programador
      por m\'{i} en Python y Java.\\
      Este es un tema interesante ya que a trav\'{e}s de simples
      l\'{i}neas de c\'{o}digo es posible que un ordenador, el
      cual solo entiende ceros y unos, pueda predecir situaciones
      complejas e incluso aprender a realizar actividades con un rendimiento
      superior al de cualquier otro humano. \\
        
      A d\'{i}a de hoy existen diversas inteligencias artificiales
      construidas a base de redes neuronales y con algoritmos que se
      explicar\'{a}n a continuaci\'{o}n, capaces de relizar tareas que
      resultan futuristas. Un ejemplo de esto es la tecnolog\'{i}a 
      desarrollada por la empresa de Elon Musk, OpenAI, capaz
      de simular redacciones humanas. Este modelo cuenta con al menos
      $175.000$ millones de par\'{a}metros de aprendizaje de acuerdo con 
      la web oficial de OpenAI. Esta es una red neuronal 
      fascinante ya que puede realizar distintas tareas
      igual que un humano.\\
      Asimismo, esta empresa ha desarrollado una gran 
      variedad de inteligencias artificiales distintas.
      Otro proyecto es una red neuronal que aprende a 
      jugar al escondite. Se trata de unos agentes 
      de color azul cuyo objetivo es, a partir de 
      distintos materiales que hay en el mapa, usarlos
      para esconderse de otros agentes. La red neuronal
      aprendi\'{o} a bases de millones de iteraciones
      a cumplir su opbjetivo de la forma m\'{a}s conveniente 
      y efectiva. Tuvo resultados impresionantes e incluso, 
      aprovechaban "bugs" del juego a fin de conseguir
      escapar o encontrar al rival.\\

      Debido al auge de las redes neuronales y todos los
      avances realizados durante la \'{u}ltima d\'{e}cada
      me ha interesado bastante este tema. Por esa raz\'{o}
      estudi\'{e} toda la base matem\'{a}tica y asimismo, 
      con muchas horas de trabajo y frustraci\'{o}n, logr\'{e}
      crear mi propia inteligencia artifical. Esta es capaz
      de aprender el patr\'{o}n de una funci\'{o}n normalizada
      y predecir el siguiente valor que tomar\'{a} en funci\'{o}n
      de los datos. Adem\'{a}s tambi\'{e}n tiene la capacidad de
      aprender una tabla de XOR, AND, OR y otras compuertas l\'{o}gicas
      con el fin de predecir. Tambi\'{e}n intent\'{e} que la red
      neuronal pudiese aprender datos reales estad\'{i}sticos y predecirlos.
      Al principio, no tuve mucho \'{e}xito, sin embargo, mediante 
      incontables iteraciones y, la carga de nuevos datos que alimentaban y 
      mejoraban cada vez m\'{a}s a la red, se obtuvieron predicciones
      precisas. \\
      Por tanto, el objetivo de este documento es explicar las bases
      matem\'{a}ticas que intervienen en la inteligencia artifical con 
      el fin de realizar predicciones. Sin embargo, es
      importante recordar que las redes neuronales no solo predicen, si no 
      que tambi\'{e}n son capaces de realizar tareas mediante el aprendizaje
      de patrones. En este caso, la tarea es predecir. \\


      \section*{¿Qué es una red neuronal?}

      



        
        
     
    
      
        

        

        
        
%
     %   \begin{table}[]
       %     \centering
      %      \begin{tabular}{rrrrr}
      %      \multicolumn{1}{c}{neurona} & \multicolumn{1}{c}{peso} & \multicolumn{1}{c}{umbral} & \multicolumn{1}{c}{capa} & \multicolumn{1}{c}{extra xd} \\
     %       1                           & 1                        & 1                          & 1                        & 2                            \\
     %       2                           & 1                        & 1                          & 1                        & 2                            \\
     %       3                           & 2                        & 3                          & 2                        & 2                           
      %      \end{tabular}
     %   \end{table}
   \end{onehalfspace}


\end{document}